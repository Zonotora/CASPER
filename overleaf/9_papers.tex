\subsection{Treehouse \cite{andersonTreehouseCaseCarbonAware2022}} 

As the effects of Denning scaling and Moore's law are coming to an end, datacenters are putting increasing constraints on the energy usage that will if not mitigated lead us in on an unsustainable path. 

\textbf{Keywords: } Service-level agreements (SLAs)
\begin{itemize}
    \item Substantial reductions in carbon intensity of datacenter computing are possible with a \textbf{software-centric} approach. This could be accomplished by making the energy and carbon visible in the application layer.
    \item Research agenda to reduce the carbon footprint of datacenter computing (software bloat, interchangeable compute, interchangeable memory, energy-aware scheduling)
\end{itemize}
    
    

\subsection{CADRE \cite{xuCADRECarbonAwareData2015}}
CADRE is a carbon-aware data replication approach that forecasts emission rates at each site and replicates data to sites to reduce the overall carbon footprint. It makes decisions online (as data are created --- avoids emissions caused by moving data frequently due to changing emission rates).

\textbf{Keywords: } footprint-replication curve, multiple-choice secretary algorithm

\begin{itemize}
    \item Focus is distributed.
\end{itemize}

\subsection{Geographical load shifting \cite{lindbergGuideReducingCarbon2021}}
Focuses on how data centers can use their geographical load flexibility to reduce the carbon emissions through clever interactions with the electricity market. This paper is quite dense.

\textbf{Keywords: }

\begin{itemize}
    \item Proposes an improved metric to guide geographic load shifting (locational marginal carbon emissions $\lambda_\text{CO2}$)
    \item Compares their improved with three other metrics.
\end{itemize}

\subsection{Carbon Explorer \cite{acunCarbonExplorerHolistic2022}}
This paper introduces a framework called, Carbon Explorer that analyzes the multi-dimensional solution space by taking into account operational and embodied carbon footprint to help datacenters operate on renewable energy 24/7.

\textbf{Keywords: }operational carbon footprint, embodied carbon footprint, service level objectives (SLOs), power purchase agreement (PPA), carbon-aware scheduling (CAS)

\begin{itemize}
    \item Aims to be 100\% Carbon free 24/7
\end{itemize}

\subsection{Enabling sustainable clouds \cite{bashirEnablingSustainableClouds2021}}
The current approach to enabling sustainable clouds focuses on improving the energy-efficiency and purchasing carbon offsets. These approaches have many limitations. This vision paper advocates ``carbon first'' by arguing that cloud platforms should virtualize the energy system by exposing the visibility to the application layer --- enabling developers to make their own abstractions for managing energy.

\textbf{Keywords: } 

\begin{itemize}
    \item 
\end{itemize}

\subsection{Let's wait awhile \cite{wiesnerLetWaitAwhile2021}}
Carbon intensity fluctuates over time. Exploiting this variability is a necessity for reducing the overall carbon emissions causes by data centers. Previous work focus mostly on carbon-aware workload migration across geo-distributed data centers. This paper investigates the potential impact of shifting computational workloads towards times where the energy supply is expected to be less carbon-intensive.