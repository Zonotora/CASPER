To formalize the scheduling and provisioning problems we device two mixed-integer quadratically constrained programs (MIQCP) as seen in Eq. \ref{provEq}-\ref{casEq}.
The server provisioner uses a carbon and request forecast for the next hour for the regions.

In the following problem definition all indices are in the set of regions, $\mathcal{R}$. We let $\bar{x}\in\{\,0,1\,\}$ be a indicator variable for when no requests are sent to a region $j$, i.e. when $\sum_j x_{ij}=0$.

\subsection{Decision variables}
\begin{equation*}
\begin{aligned}
x_{ij} &:= \text{Requests from region $i$ to region $j$} \\
s_i &:= \text{Number of servers in region $i$}
\end{aligned}
\end{equation*}


\subsection{Parameters}
\begin{equation*}
\begin{aligned}
I_j &:= \text{Carbon intensity in region $j$}\\
\lambda_i &:= \text{Request rate from region $i$}\\
\ell_{ij} &:= \text{Latency from region $i$ to $j$}  \\
c_j &:= \text{Capacity for each server in region } j \\
L_{i} &:= \text{Maximum tolerated latency from region $i$}  \\
K &:= \text{Maximum number of servers } \\
\end{aligned}
\end{equation*}

\subsection{Carbon Aware Provisioning Model}
\label{sec:capm}

\begin{subequations}
    \begin{align}
    \underset{x}{\min} \quad & \sum_j I_j\sum_i x_{ij} \\
    \text{s.t.}  \quad & \sum_i x_{ij} \leq s_jc_j \\
    & \sum_j s_j \leq K \\
    & x_{ij}\left(\ell_{ij}-L_i\right) \leq 0 \\
    & \sum_j x_{ij} = \lambda_i \\
    & \bar{x}_js_j=0 \\
    & x_{ij},s_j \in\mathbb{Z}_{\geq 0}
    \end{align}
    \label{provEq}
\end{subequations}


\subsection{CAS Model}
    \begin{subequations}
    \begin{align}
    \underset{x}{\min} \quad & \sum_j I_j\sum_i x_{ij} \\
    \text{s.t.}  \quad & \sum_i x_{ij} \leq s_jc_j \\
    & x_{ij}\left(\ell_{ij}-L_i\right) \leq 0 \\
    & \sum_j x_{ij} = \lambda_i \\
    & \bar{x}_js_j=0 \\
    & x_{ij},s_j \in\mathbb{Z}_{\geq 0}
    \end{align}
    \label{casEq}
\end{subequations}

To solve the two MILPs we use the library \texttt{PuLP} which is a Python interface to the Coin-or branch and cut (CBC) solver. the relaxed MILP is solved quickly. 
%The integer constraint can be relaxed, so the problem becomes a simple LP.
To solve minimization problems \ref{casEq} and \ref{provEq} we relax them to MILP problems by removing the quadratic constraints 1f and 2e. 


%$$
%    \begin{align}
%    \underset{x}{\min} \quad & \sum_j I_j\sum_i x_{ij} & \text{(1a)}\\
%    \text{s.t.}  \quad & \sum_i x_{ij} \leq s_jc_j & \text{(1b)}\\
%    & x_{ij}\left(\ell_{ij}-L_i\right) \leq 0 & \text{(1c)} \\
%    & \sum_j x_{ij} = \lambda_i & \text{(1d)}\\
%    & \bar{x}_j s_j=0 & \text{(1e)}\\
%    & x_{ij} ,s_j \in\mathbb{Z}_{\geq 0}& \text{(1f)}
%    \end{align}
%$$
